\documentclass{beamer}
\usepackage[utf8]{inputenc} % inputenc for encoding to utf8

\usepackage{standalone}
\usepackage{tikz}
\usepackage{xcolor}

\usepackage{multicol}
\usepackage{amsmath}

\usetheme[progressbar=frametitle]{metropolis}
\setbeamertemplate{frame numbering}[fraction]
\usefonttheme{serif}
\useoutertheme{metropolis}
\useinnertheme{metropolis}
\usefonttheme{metropolis}
%\setbeamercolor{background canvas}{bg=white}

\definecolor{mygreen}{rgb}{.125,.5,.25}
\definecolor{bittersweet}{rgb}{1.0, 0.44, 0.37}
\definecolor{amethyst}{rgb}{0.6, 0.4, 0.8}
%\usecolortheme[named=mygreen]{structure}

\usecolortheme{spruce}

\title[Complex analysis]{Topics in complex analysis}
\subtitle{FTA and M\"{o}bius transformations}
\author{Ryan Nguyen}
\institute{}
\date{}

\begin{document}
\metroset{block=fill}

\begin{frame}
\titlepage
\end{frame}

\begin{frame}[t]{Complex analysis} \vspace{4pt}
\textbf{complex analysis}: an extension of real analysis to calculus with complex variables
\begin{itemize}
	\item define differentiation by the difference quotient
	\item define integration by the Riemann sum
\end{itemize}
\textbf{holomorphic}: complex differentiable.
\begin{itemize}
	\item analytic
	\item infinitely complex differentiable
\end{itemize}
\end{frame}

\begin{frame}[t]{Some results} \vspace{3pt}
\begin{block}{Cauchy's integral formula}
Let \(U\) be an open subset of the complex numbers. If \(f: \mathbf{C} \rightarrow \mathbf{C}\) is holomorphic on \(U\), then we have
\[
f(a) = \frac{1}{2\pi i} \oint \frac{f(z)}{z-a} dz,
\]
integrating in a counter-clockwise loop around \(a \in U\), and similarly for higher order derivatives.
\end{block}
A corollary involving the \(1\)st derivative is Liouville's theorem.
\begin{block}{Liouville's theorem}
Let \(f: \mathbf{C} \rightarrow \mathbf{C}\) be an entire function, i.e., holomorphic on the entire complex plane, and bounded. Then \(f\) is constant.
\end{block}
\end{frame}

\begin{frame}[t]{Fundamental theorem of algebra (Liouville)} \vspace{4pt}
\begin{block}{Fundamental theorem of algebra}
Every polynomial of degree at least \(1\) has a zero.
\end{block}
\textit{Idea.} Using Liouville's theorem, force a contradiction. Let \(p\) be a non-constant polynomial with complex coefficients.
\begin{enumerate}
	\item Suppose no zeros. Define \(f(z) := \frac{1}{p(z)}\).
	\item Notice that \(f\) is entire.
	\item Notice that \(f\) is bounded.
	\item Then by Liouville \(f\) must then be constant, implying that \(p\) is constant---a contradiction.
\end{enumerate}
\end{frame}

\begin{frame}[b]{Fundamental theorem of algebra (Liouville)}
\begin{figure}
  \includestandalone[width=\textwidth]{boundedness}
  \caption{Demonstration of boundedness of \(f(z) := \frac{1}{p(z)}\)}
  \label{fig:boundedness}
\end{figure}
\end{frame}

\begin{frame}[t]{Fundamental theorem of algebra (Picard)} \vspace{4pt}
\begin{block}{Picard's little theorem}
A non-constant entire function omits at most two points from its range.
\end{block}
\textit{Idea.} Force a contradiction by explicitly constructing the root.
\begin{enumerate}
	\item Suppose \(p\) has no roots, i.e., it excludes \(0\) from its range. Then its range must be exactly \(\mathbf{C} \setminus \{0\}\).
	\item Then \(p(\mathbf{C})\) contains \(\frac{1}{2}, \frac{1}{3}, \frac{1}{4},\) etc., points nearer and nearer \(0\).
	\item Pull corresponding points in the domain to construct a sequence converging to a root.
\end{enumerate}
\end{frame}

\begin{frame}[t]{Fundamental theorem of algebra (Picard)}
\begin{figure}
	% resize because ... it's too big
  	\includestandalone[width=0.7\textwidth]{construction}
  	\caption{Construction of a root of \(p\)}
  	\label{fig:consruction}
\end{figure}
\end{frame}

\begin{frame}[t]{M\"{o}bius transformations} \vspace{4pt}
Let \(M\) denote the set of all functions \(z \mapsto \frac{az + b}{cz + d}, \, ad - bc \ne 0\).
\begin{itemize}
	\item Each function of \(M\) is a conformal mapping
	\item \(M\) forms a group under function composition \(\circ\)
	\item This group is homomorphic to the set of \(2 \times 2\) matrices under matrix multiplication:

	\[
		\begin{pmatrix}a & b \\ c & d\end{pmatrix} \begin{pmatrix}e & f \\ g & h\end{pmatrix} = \phi \circ \psi
	\]
	where \(\phi(z) := \frac{az + b}{cz + d}\) and \(\psi(z) := \frac{ez + f}{gz + h}\).
	\item M\"{o}bius transformations map circles to circles or lines, and they map lines to circles or lines
\end{itemize}
\end{frame}

\begin{frame}[t]{M\"{o}bius transformations} \vspace{4pt}
Circles are loops. Lines are just loops that include \(\infty\)!

\textit{Idea.} All M\"{o}bius transformations \(z \mapsto \frac{az + b}{cz + d}\) are a composition of scales/rotations (\(z \mapsto kz\)), translations (\(z \mapsto z + z_0\)), and inversions (\(z \mapsto \frac{1}{z}\)). Inversions are interesting.

(some illustration of what happens to lines/circles wrt the hole at the origin)
\end{frame}

\begin{frame}[standout]
\flushleft
Conclusion
\end{frame}

\end{document}