\documentclass{beamer}
\usepackage[utf8]{inputenc} % inputenc for encoding to utf8

\usepackage{multicol}

\usetheme[progressbar=frametitle]{metropolis}
\setbeamertemplate{frame numbering}[fraction]
\usefonttheme{serif}
\useoutertheme{metropolis}
\useinnertheme{metropolis}
\usefonttheme{metropolis}
%\setbeamercolor{background canvas}{bg=white}

\definecolor{mygreen}{rgb}{.125,.5,.25}
\definecolor{bittersweet}{rgb}{1.0, 0.44, 0.37}
\definecolor{amethyst}{rgb}{0.6, 0.4, 0.8}
%\usecolortheme[named=mygreen]{structure}

\usecolortheme{spruce}

\title[Complex analysis]{Topics in complex analysis}
\subtitle{FTA and M\"{o}bius transformations}
\author{Ryan Nguyen}
\institute{}
\date{}

\begin{document}
\metroset{block=fill}

\begin{frame}
\titlepage
\end{frame}

\begin{frame}[t]{Complex analysis?} \vspace{4pt}
\textbf{complex analysis}: an extension of real analysis to calculus with complex variables
\begin{itemize}
	\item define differentiation by the difference quotient
	\item define integration by the Riemann sum
\end{itemize}
\textbf{holomorphic}: complex differentiable.
\begin{itemize}
	\item infinitely complex differentiable!
	\item analytic (partially follows from former point)
\end{itemize}
\end{frame}

\begin{frame}[t]{Some results} \vspace{4pt}
\begin{block}{Cauchy's integral formula}
Let \(U\) be an open subset of the complex numbers. If \(f: \mathbf{C} \rightarrow \mathbf{C}\) is holomorphic on \(U\), then we have
\begin{equation*}
f(a) = \frac{1}{2\pi i} \oint \frac{f(z)}{z-a} dz,
\end{equation*}
integrating in a counter-clockwise loop around \(a \in U\), and similarly for higher order derivatives.
\end{block}
A corollary involving the \(1\)st derivative is Liouville's theorem.
\begin{block}{Liouville's theorem}
Let \(f: \mathbf{C} \rightarrow \mathbf{C}\) be an entire function, i.e., holomorphic on the entire complex plane, and bounded. Then \(f\) is constant.
\end{block}
\end{frame}

\begin{frame}[t]{Fundamental theorem of algebra (Liouville)} \vspace{4pt}
\textit{Idea.} Using Liouville's theorem, force a contradiction. Let \(p\) be a non-constant polynomial with complex coefficients.
\begin{enumerate}
	\item Suppose no zeros. Define \(f(z) = \frac{1}{p(z)}\).
	\item Notice that \(f\) is entire.
	\item Notice that \(f\) is bounded.
	\item Then by Liouville \(f\) must then be constant, implying that \(p\) is constant---a contradiction.
\end{enumerate}
\end{frame}

\begin{frame}[t]{Fundamental theorem of algebra (Picard)} \vspace{4pt}
\begin{block}{Picard's little theorem}
A non-constant entire function omits at most two points from its range.
\end{block}
\textit{Idea.} Force a contradiction by explicitly constructing the root.
\begin{enumerate}
	\item Suppose \(p\) has no roots, i.e., it excludes \(0\) from its range. Then its range must be exactly \(\mathbf{C} \setminus \{0\}\).
	\item Then \(p(\mathbf{C})\) contains \(1/2, 1/3, 1/4,\) etc., points nearer and nearer \(0\).
	\item Pull corresponding points in the domain to construct a sequence converging to a root.
\end{enumerate}
\end{frame}

\begin{frame}[standout]
\flushleft
Conclusion
\end{frame}

\end{document}